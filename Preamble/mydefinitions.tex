%-------------------------------------------------------------------------------
%	VALUES FOR THE THESIS
%-------------------------------------------------------------------------------

\newcommand{\name}{James Schloss} % Author name
\newcommand{\thesistitle}{Massively parallel split-step Fourier techniques for simulating quantum systems on graphics processing units} % Title of the thesis
\newcommand{\submissiondate}{October, 2019} % Submission date "Month, year"
\newcommand{\supervisor}{Thomas Busch} % Supervisor name
%\newcommand{\cosupervisor}{C.~O'Supervisor} % Co-Supervisor name, comment this line if there is none


%-------------------------------------------------------------------------------
%	BIBLIOGRAPHY STYLE (pick the style you want)
%-------------------------------------------------------------------------------

\usepackage[square, numbers, sort&compress]{natbib} % for bibliography - Square brackets, citing references with numbers, citations sorted by appearance in the text and compressed (as in [4-7])
\usepackage{amsmath}
%\usepackage[longnamesfirst,round]{natbib} % Natural Sciences bibliography

%\bibliographystyle{Preamble/physics_bibstyle} % You may use a different style adapted to your field
%\bibliographystyle{abbrvnat} % You may use a different style adapted to your field
\bibliographystyle{unsrtnat} % You may use a different style adapted to your field


%-------------------------------------------------------------------------------
%	YOUR PACKAGES (be careful of package interaction)
%-------------------------------------------------------------------------------

\usepackage{amsthm,amsmath,amssymb,amsfonts,bbm, tikz}% Math symbols
\usepackage{subfigure}

%-------------------------------------------------------------------------------
%	YOUR DEFINITIONS AND COMMANDS
%-------------------------------------------------------------------------------

% New Commands
\newcommand{\bea}{\begin{eqnarray}} % Shortcut for equation arrays
\newcommand{\eea}{\end{eqnarray}}
\newcommand{\e}[1]{\times 10^{#1}}  % Powers of 10 notation

% Defining a theorem box for Criteria
\newtheorem{critere}{Criterion}
\newcommand{\crit}[2]{
\begin{center}  
\fbox{ \begin{minipage}[c]{0.9 \textwidth}
\begin{critere}
\textbf{\textup{ #1}} --- #2
\end{critere}
\end{minipage}  } \end{center}
}

\usepackage{listings}

\lstdefinestyle{c++}{
  belowcaptionskip=1\baselineskip,
  breaklines=true,
  frame=single,
  xleftmargin=\parindent,
  language=C,
  showstringspaces=false,
  basicstyle=\footnotesize\ttfamily,
  keywordstyle=\bfseries\color{green!40!black},
  commentstyle=\itshape\color{purple!40!black},
  identifierstyle=\color{blue},
  stringstyle=\color{orange},
  numbers=left
}

\lstset{language=C,
        basicstyle=\ttfamily,
        keywordstyle=\color{blue}\ttfamily,
        stringstyle=\color{red}\ttfamily,
        commentstyle=\color{green}\ttfamily,
        morecomment=[l][\color{magenta}]{\#},
        basicstyle=\footnotesize\ttfamily,
        showstringspaces=false
}

\lstdefinelanguage{julia}
{
  keywordsprefix=\@,
  morekeywords={
    exit,whos,edit,load,is,isa,isequal,typeof,tuple,ntuple,uid,hash,finalizer,convert,promote,
    subtype,typemin,typemax,realmin,realmax,sizeof,eps,promote_type,method_exists,applicable,
    invoke,dlopen,dlsym,system,error,throw,assert,new,Inf,Nan,pi,im,begin,while,for,in,return,
    break,continue,macro,quote,let,if,elseif,else,try,catch,end,bitstype,ccall,do,using,module,
    import,export,importall,baremodule,immutable,local,global,const,Bool,Int,Int8,Int16,Int32,
    Int64,Uint,Uint8,Uint16,Uint32,Uint64,Float32,Float64,Complex64,Complex128,Any,Nothing,None,
    function,type,typealias,abstract
  },
  sensitive=true,
  morecomment=[l]{\#},
  morestring=[b]',
  morestring=[b]" 
}

\lstdefinestyle{julia}{
  belowcaptionskip=1\baselineskip,
  breaklines=true,
  xleftmargin=\parindent,
  language=Julia,
  showstringspaces=false,
  basicstyle=\footnotesize\ttfamily,
  keywordstyle=\bfseries\color{green!40!black},
  commentstyle=\itshape\color{purple!40!black},
  identifierstyle=\color{blue},
  stringstyle=\color{orange},
}

