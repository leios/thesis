%%%%%%%%%%%%%%%%%%%%%%%%%%%%%%%%%%%%%%%%%
% Beamer Presentation
% LaTeX Template
% Version 1.0 (10/11/12)
%
% This template has been downloaded from:
% http://www.LaTeXTemplates.com
%
% License:
% CC BY-NC-SA 3.0 (http://creativecommons.org/licenses/by-nc-sa/3.0/)
%
% Modified by Jeremie Gillet in November 2015 to make an OIST Skill Pill template
%
%%%%%%%%%%%%%%%%%%%%%%%%%%%%%%%%%%%%%%%%%

%----------------------------------------------------------------------------------------
%	PACKAGES AND THEMES
%----------------------------------------------------------------------------------------

\documentclass{beamer}

\mode<presentation> {

\usetheme{}

\definecolor{OISTcolor}{rgb}{0.65,0.16,0.16}
\usecolortheme[named=OISTcolor]{structure}

%\setbeamertemplate{footline} % To remove the footer line in all slides uncomment this line
%\setbeamertemplate{footline}[page number] % To replace the footer line in all slides with a simple slide count uncomment this line

\setbeamertemplate{navigation symbols}{} % To remove the navigation symbols from the bottom of all slides uncomment this line
}

\usepackage{graphicx} % Allows including images
\usepackage{booktabs} % Allows the use of \toprule, \midrule and \bottomrule in tables
\usepackage{textpos} % Use for positioning the Skill Pill logo
\usepackage{fancyvrb}
\usepackage{tikz}
\usepackage{hyperref}
\usepackage{listings}

\definecolor{dkgreen}{rgb}{0,0.6,0}
\definecolor{gray}{rgb}{0.5,0.5,0.5}
\definecolor{mauve}{rgb}{0.58,0,0.82}
\setbeamertemplate{frametitle}[default][center]

\lstset{frame=tb,
  language=python,
  aboveskip=3mm,
  belowskip=3mm,
  showstringspaces=false,
  columns=flexible,
  basicstyle={\small\ttfamily},
  numbers=none,
  numberstyle=\tiny\color{gray},
  keywordstyle=\color{blue},
  commentstyle=\color{dkgreen},
  stringstyle=\color{mauve},
  breaklines=true,
  breakatwhitespace=true,
  tabsize=3
}

%----------------------------------------------------------------------------------------
%	TITLE PAGE
%----------------------------------------------------------------------------------------

\title[PhD presentation]{Massively parallel split step Fourier techniques for simulating quantum systems} % The short title appears at the bottom of every slide, the full title is only on the title page

\author{James Schloss} % Your name
\institute[OIST] % Your institution as it will appear on the bottom of every slide, may be shorthand to save space
{
Okinawa Institute of Science and Technology \\ % Your institution for the title page
\textit{james.schloss@oist.jp}
}
\date{December 9, 2019} % Date, can be changed to a custom date

\begin{document}

\setbeamertemplate{background}{\includegraphics[width=\paperwidth]{OISTBG.png}} % Adding the background logo

\begin{frame}
\vspace*{1.4cm}
\titlepage % Print the title page as the first slide
\end{frame}


\setbeamertemplate{background}{} % No background logo after title frame

\addtobeamertemplate{frametitle}{}{% Adding the Skill Pill logo on the title screen after title frame
\begin{textblock*}{100mm}(.92\textwidth,-0.75cm)
\includegraphics[height=1cm]{OIST}
\end{textblock*}}

\begin{frame}
\frametitle{Overview}
This project has created fast, GPU-accelerated software for the simulation of superfluid systems
\begin{columns}
\column{0.7\textwidth}
\begin{itemize}
\onslide<2->
\item The split-step Fourier method
\onslide<3->
\item Non-adiabatic superposition states via quantum state engineering
\onslide<4->
\item GPU architecture and the GPUE codebase
\onslide<5->
\item Chaotic vortex dynamics in a few-vortex system
\onslide<6->
\item Vortex rings with artificial magnetic fields
\onslide<7->
\item Conclusions and future directions
\end{itemize}
\column{0.3\textwidth}
\onslide<1->
\includegraphics<1>[width=\textwidth]{GPUE.png}
\includegraphics<2>[width=\textwidth]{../data/splitop/method/split_op_method.pdf}
\includegraphics<3>[width=\textwidth]{../data/1d/scheme.pdf}
\includegraphics<4>[width=\textwidth]{../data/gpu/gputhreads.pdf}
\includegraphics<5>[width=\textwidth]{../data/2d/evolution/evolution.pdf}
\includegraphics<6>[width=\textwidth]{../data/3d/HE21_3d.png}
\end{columns}
\end{frame}

\begin{frame}
\center \huge The split-step Fourier method
\end{frame}

\begin{frame}
\frametitle{Heisenberg uncertainty principle}
\end{frame}

\begin{frame}
\frametitle{The Fourier Transform}
\end{frame}

\begin{frame}
\frametitle{Cooley-Tukey}
\end{frame}

\begin{frame}
\frametitle{Hamiltonian}
\end{frame}

\begin{frame}
\frametitle{Bose-Einstein condensation}
\end{frame}

\begin{frame}
\frametitle{Superfluid rotation}
\end{frame}

\end{document}
