\chapter{Introduction to the GPUE codebase for $n$-dimensional simulations of quantum systems on the GPU}
\label{ch:gpue}

This chapter introduces GPUE, the GPU-based Gross-Pitaevskii Equation solver and all features necessaru for simulations of $n$-dimensional quantum systems to be used in the following chapters.

\subsection{Parallel summation for renormalization and energy calculation}

As mentioned in Chapter~\ref{ch:splitop}, the normalization routine is essential for a proper evolution in imaginary time, and renormalization must occur freqeuntly enough to prevent numerical overflow or underflow errors.
For safety in GPUE, we do the renormalization every imaginary step; however, this comes at a cost to performance as the renormalization necessitates a parallel summation over the wavefunction.
Additionally, the user may choose to calculate the energy of the atomic system at every timestep, which requires both a parallel summation and a large memory reseviour.


\subsection{cuFFT library}

Here, discuss the 1D FFT's over $n$-dimensional data

\section{Dynamic field input and output in GPUE}

As mentioned in Chapter~\ref{ch:1d}, quantum engineering typically requires some form of time-dependent variables, along with evolution in real time.
This means that the user must be able to input a time-dependent equation and must be able to read in a time-dependent field of their choosing and read this out in an efficient storage format.
Because we chose to write GPUE in CUDA, there is no straightforward method for the user to input time-dependent fields without recompiling the source code and modifying CUDA kernels at will, which is unnecessarily cumbersome.
As such, we have provided a method for users to input the fields of their choosing as strings, which will be transpiled into an array of operations to perform on the GPU through Abstract Syntax Trees (ASTs).

For fileIO, we had initially considered implementing the Compressed Split-Step Fourier Method (CSSFM)~\ref{bayindir2015}; however, this was shown to be unsuitable for the case of $n$-dimensional systems with vortex dynamics.
Instead, we simply implemented HDF5, and we will discuss this implementation here.

\subsection{Abstract syntax trees for dynamic fields and memory management}

\subsection{Implementation of HDF5 filesystem for fileIO}

\section{Vortex highlighting via Sobel filter}

As described in Chapter~\ref{ch:dynamics}, vortex tracking in two dimensions is not always straghtforward for non-harmonic traps.
In three dimensions, vortices are no longer confined to a plane and can extend in any direction, so long as the vortex lines either end at the end of the superfluid or reconnect in the form of vortex rings or more complicated vortex structures.
This is a much more difficult problem which does not have many solutions in superfluid simulations where the superfluid does not fill the simulation domain.

The current state-of-the-art solution has been proposed by Villois \textit{et. al}, and required finding density dips in the superfluid as initial guesses as to where a vortes might exist.
From there, a vorticity plane is determined and the entire vortex is discovered by moving perpendicularly to the vorticity plane at each gridpoint.
This is a tedious and time-consuming process that does not lend itself well to GPGPU computation without largescale communication between the host and device, which is not recommended.
As such, we are currently seeking a more computationally efficient method for tracking vortices in three dimensions, and a possible method will be further discussed, but not tested in the conclusion of this work.

As our system do not necessarily fill the contents of our simulation domain, the proposed method will not work without some modification.
We could still use the method if we have some understanding of the trapping geometry; however, this is not always the case, as we will see in a future example simulation in Chapter~\ref{ch:vortex_states}.

As such, instead of focusing on vortex \textit{tracking}, we have instead implemented a simple vortex \textit{highlighting} scheme for three dimensions.
This is simply a Sobel filter on the condensate density, and can easily create crisp visualizations like those found in the computer graphics literature~\cite{guo2018}.

\jrs{describe edge detection, maybe canny?}

This is a difficult problem that required further study; however, vortex highlighting is enough for most three dimensional vortex simulations.
A possible method to track vortices by using a sobel filter can be found in the conclusion.

\section{CuFFT routine for gauge field simulation}

\section{DistributedTranspose.jl}
While working on CuFFT methods for applying gauge fields to the superfluid simulation, we also discussed several smallscale optimizations that could be done.

\begin{itemize}
\item{FFT needs coalesced data}
\item{Transposes allow for this}
\item{Distributed Transposes could allow for multi-GPU simulations}
\end{itemize}

\subsection{Unit tests available}

\section{CSSFM implentation and future direction}
