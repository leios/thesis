\chapter{Vortex analysis of two-dimensional superfluid systems}
\label{ch:2d}

In a similar fashion to Chapter~\ref{ch-1d}, this chapter will start with a disclaimer about the dimensionality.
In principle, all real-world physics is three-dimensional, but in a similar fashion to how a one dimensional cigar-shaped BEC can be created, a pancake-like geometry can also be constructed by increasing the trapping in the $z$ direction with respect to the $x$ and $y$ directions.

This geometry has the advantage of displaying the effects of angular momentum on a superfluid system in the most intuitive way.
As mentioned in Chapter~\ref{ch-introduction}, the angular momentum around vortices in a superfluid is quantized, which means that as we rotate a superfluid faster, vortices will enter the system one at a time.
As more and more vortices enter the system, they will eventually pack the superfluid as tightly as possible and create a triangular Abrikosov lattice, that is well-known from superconductor physics.
When the rotational frequency $\Omega$ exceeds the trap frequency $\omega$, the superfluid will no longer be contained by the trap, but large vortex latices can be created by rotating sufficiently close to this bound [CITE LEE].

This chapter will introduce the methods for vortex tracking in GPUE, along with methods for engineering specific vortex configurations via phase imprinting.
We will also discuss the analysis of vortex motion via Lyapunov exponents with an example of chaotic motion being generated in few-body vortex systems for rotating BECs.


\section{Phase imprinting implementation in GPUE}
\section{Vortex tracking implemetation in 2D for GPUE}
\section{The Lyapunov exponent}

\section{Chaotic few-body vortex dynamics in rotating Bose--Einstein condensates}
