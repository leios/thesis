\chapter{Introduction to the split-operator method for vortex simulations}
\label{ch:splitop}

The split-operator method is a pseudo-spectral technique for solving partial differential equations and is particularly useful for several nonlinear partial differential equations such as the Gross--Pitaevskii Equation (GPE),

$$
i \hbar \frac{\partial \Psi(\mathbf{r},t)}{\partial t} = \left(\frac{p^2}{2m} + V_0 + g |\Psi(\mathbf{r},t)|^2 \right)\Psi(\mathbf{r},t)
$$

where $\mathbf{r}$ is a position vector, $\Psi(\mathbf{r},t)$ is a many-body quantum wavefunction, $p = -i\hbar\nabla$ is the momentum operator, $V_0$ is the trapping potential, g = $\frac{4\pi\hbar^2 a_s}{m}$, $a_s$ is the scattering length of the atomic species, and $m$ is the mass.

By solving the GPE with time, we can determine the dynamics of Bose--Einstein Condensates (BECs).
The states found with the GPE are virtually identical to those found in experiments.

One important technique for solving the GPE is the pseudospectral split-operator method, which requires splitting the hamiltonian into separate operators in position and momentum space and operating on the wavefunction in the appropriate space.

\section{The split-operator method}

As mentioned, the split-operator method splits the Hamiltonian into separate operators and uses the Fourier transform to ensure that these operators work on the wavefunction in the appropriate space.

First, we assume a somewhat general solution to our wavefunction,

$$
\Psi(\mathbf{r},t + dt) = \left[e^{-\frac{i\hat{H}dt}{\hbar}}\right]\Psi(\mathbf{r},t) = \left[e^{-\frac{i(\hat{H}_r + \hat{H}_k)dt}{\hbar}}\right]\Psi(\mathbf{r},t)
$$

and assume we are simulating our system by a series of small timesteps ($dt$), we can perform similar splitting by using the Baker-Campbell-Housdorff formula:

$$
\Psi(\mathbf{r},t+dt) = \left[e^{-\frac{i\hat{H}_rdt}{\hbar}}e^{-\frac{i\hat{H}_kdt}{\hbar}}e^{-\frac{[i\hat{H}_r, i\hat{H}_k]dt^2}{2}}\right]\Psi(\mathbf{r},t)
$$

This accrues a small amount of error ($dt^2$) related to the commutation of the real and momentum-space components of the Hamiltonian, which is a noticeably high.
In order to change the $dt^2$ error to $dt^3$, we can split the system by performing a half-step in position space before doing a full-step in momentum space, through a process called \textit{Strang Splitting} like so:

$$
\Psi(\mathbf{r},t+dt) = \left[e^{-\frac{i\hat{H}_rdt}{2\hbar}}e^{-\frac{i\hat{H}_kdt}{\hbar}}e^{-\frac{i\hat{H}_rdt}{2\hbar}} \right]\Psi(\mathbf{r},t) + \mathcal{O}(dt^3)
$$

We can then address each part of this solution in chunks, first in position space, then in momentum space, then in position space again by using Fourier Transforms.
Which looks something like this:

$$
\Psi(\mathcal{r}, t+dt) = \left[\hat{U}_r\left(\frac{dt}{2}\right)\mathcal{F}^{-1}\left[\hat{U}_k(dt) \mathcal{F} \left[\hat{U}_r\left(\frac{dt}{2}\right) \Psi(\mathbf{r},t) \right] \right] \right] + \mathcal{O}(dt^3)
$$

where $\hat{U}_r = e^{-\frac{i\hat{H}_rdt}{\hbar}}$, $\hat{U}_k = e^{-\frac{i\hat{H}_kdt}{\hbar}}$, and $\mathcal{F}$ and $\mathcal{F}^{-1}$ indicate forward and inverse Fourier Transforms.
A flowchart of how we perform this operation can be found in FIGURE, and pseudo-code can be found in LISTING (CHANGE ENUMERATE)

\begin{enumerate}
\item Multiply the wavefunction in real space with the real-space operator.
\item Flip to momentum space with a Fourier transform.
\item Multiply the momentum-space wavefuntion by the momentum-space operator.
\item Flip to position space with an inverse Fourier transform.
\item Repeat 1-4 until satisfied.
\end{enumerate}

This will allow for us to simulate the dynamics of a simple quantum system.
For example, if we guess that our initial wavefunction is gaussian-like and is slightly offset from the center or the trap, this should allow us to see our wavefunction ``sloshing'' back and forth in our trap, as shown in FIGURE.

In addition, we can find the lowest energy state of our system by performing a Wick rotation and using $\tau = it$ for the simulation.
This changes the solution from a sinusoidal to an exponential decay in the energy,

$$
\Psi(\mathbf{r},\tau + d\tau) = \left[e^{-\frac{\hat{H}d\tau}{\hbar}}\right]\Psi(\mathbf{r},\tau) = \left[e^{-\frac{(\hat{H}_r + \hat{H}_k)d\tau}{\hbar}}\right]\Psi(\mathbf{r},\tau)
$$

ADD ENERGY STATE ARGUMENT

\section{Modifications to the split-operator method for superfluid vortex simulations}

To induce rotational effects in a BEC, we need to introduce a rotational operator to the Hamiltonian,

$$
\mathcal{\hat H}_{ROT} = \frac{p^2}{2m} + V_0 + g|\Psi(\mathbf{r},t)|^2 - \Omega L_z
$$

where $\Omega$ is the rotational frequency and $L_z = xp_y - yp_x$ is the angular momentum operator.
This will cause a rotation around the $z$ axis and create vortices in the BEC when rotating about some critical rotation frequency.
Notably, there is also a maximum rotation frequency at which the atoms will leave the trap, which is dependent on the frequency of the external trap.

By using this rotational operator, all vortices will align themselves along the $\hat z$ direction and there are few easy ways to create vortex structures more complicated than vortex lines.
To generate more complex vortex structures in a stable way, it is often worth implementing an artificial magnetic field by incorporating the rotation term into the $\frac{p^2}{2m}$ term like so:

$$
\mathcal{\hat H}_{ROT} = \frac{(p-m\mathbf{A})^2}{2m} + V_0 + g|\Psi(\mathbf{r},t)|^2
$$

Where $\mathbf{A}$ is a gauge field.
This creates three separate terms: $\frac{p^2}{2m}$, $\frac{\mathbf{mA}^2}(2)$, and $p\mathbf{A}$.
In the case of rotation, the final term corresponds to $L_z$.
As we will see in this text, it is possible to generate much more complicated vortex structures with this method.

\section{Numerical considerations for implementing the split-operator method}

\subsection{Numerical accuracy}

\subsection{Periodic boundary conditions}

\subsection{Comparison with other methods for simulating superfluid systems}
