\chapter{Generation, control and detection of 3D vortex structures in superfluid systems}
\label{ch:vortex_states}
\jrs{Though some of this work focuses on using the SSFM to simulate superfluid vortex states, we will not be discussing vortex-point or vortex-filament simulations, as this work focuses primarily on engineering appropriate quantums states, whilee vortex-point and vortex filament methods focus primarily on vortex structures, themselves.}

\section{Vortex ring dynamics in BEC systems}
\label{sec:vortex}

As mentioned in Section~\ref{sec:intro}, in BEC systems with large amounts of angular momentum and a single axis of rotation, the vortices will create a two-dimensional Abrikosov lattice. 
This regular structure is a direct consequence of the quantization of angular momentum in quantum mechanics.
There are many interesting features to vortices in superfluid systems, many of which follow from classical fluid dynamic theory~\ref{Fetter2009}, which is a well-studied field and covered in many texts~\cite{Faber1995, Kundu2012, Tritton1988, Landau1987}
Here, we discuss the dynamics of vortex rings in BEC systems.

In a three dimensional BEC and other superfluid systems, vortices can be described as as a system of freely-moving, current-carrying wires located along the axis of rotation. 
In this way, it is possible to model such systems using the Biot-Savart law ~\cite{Schwarz1985}
\begin{equation}
    \boldsymbol{v}_s(\boldsymbol{r},t) = \frac{\kappa}{4\pi}\int\frac{|s_1-\boldsymbol{r}|\times ds_1}{|s_1-\boldsymbol{r}|^3},
\label{eqn:BS}
\end{equation}
where $\kappa = \hbar / m$ is the circulation. 
This method of modelling a superfluid system has been called the ``vortex filament model,'' and allows us to attain an intuitive understanding for the motion of vortices in a superfluid system; however, it falls short in several areas, such as sound wave propagation and vortex reconnections~\cite{Zuccher2012}. 
In electrodynamics, current can only flow within a closed loop or between two conductors.
Similarly, when a vortex filament is generated by injecting angular momentum into a superfluid system, it must either end at the edges of the superfluid or reconnect with other vortices.

Imagine providing a superfluid with an angular momentum of $\hbar k$.
In this case, a single vortex will be generated along the axis of rotation.
If another vortex is generated, we would expect it to change the dynamics of the superfluid and move our initial vortex.
If the new vortex has an angular momentum of $+\hbar k$, the two vortices will be repelled from each other, and if the new vortex has an angular momentum of  $-\hbar k$, it will be called an ``antivortex'' and the pair will be attracted to each other, eventually annihilating on contact.
If we rotate the condensate with integer multiples of $\hbar k$ angular momentum, we can generate many vortices, all rotating in the same direction and repelling off each other, creating the Abrikosov lattice structure from their mutual repulsion~\cite{Fetter2010}.

ADD FIGURE

By modifying our axis of rotation, we may imagine more complex vortex topologies to be created. 
For example, imagine the toroidal velocity field shown in Figure~\ref{fig:V_R}, which will lead to  vortex filaments generating a ring instead of a single line. 
This vortex ring structure is common in large, three dimensional modelling of fluid systems and is a direct consequence of the required connections of vortex lines.
In other words, the stability of vortex rings is ensured by Kelvin's theorem~\cite{Donnelly1991}, which means that unstable excitations may decay into vortex rings~\cite{Anderson2001}.
Now, we will discuss the dynamics of these rings.

We will begin our discussion of vortex rings with the thin-core approximation in inviscid fluids, as described by Barenghi and Donnelly~\cite{Barenghi2009}. 
If we imagine a thin vortex ring of radius $R\gg a$, where a is the core radius, and  with circulation $\Gamma$ moving in an inviscid fluid with density $\rho$, the kinetic energy of the ring is given by
\begin{equation}
    E = \frac 1 2 \rho \Gamma^2R\left[\ln\left(\frac{8R}{a}\right)-\alpha\right],
\end{equation}
where $\alpha$ is a parameter determined by the system, and different
values of $\alpha$ correspond to different core models for vortex motion. 
For example, it is 1.615 for the GPE~\cite{Roberts1971}.
This equation follows from the Biot-Savart law (Equation~\eqref{eqn:BS}) above. 
The momentum of the system is given by
\begin{equation}
    p = \rho \Gamma \pi R^2,
\end{equation}
and the self-induced velocity is~\cite{Roberts1970}
\begin{equation}
    v = \frac{\partial E}{\partial p} = \left(\frac{\Gamma}{4 \pi R}\left[\ln(\frac{8R}{a})-\beta\right]\right),
\end{equation}
with $\beta = \alpha - 1$ under constant pressure, and $\beta = \alpha - 3/2$ under constant volume. 
This method of vortex ring analysis has been extended to rings with finite, but small values of the core radius~\cite{Fraenkel1970, Fraenkel1972}.
%\tbc{These equations come from the Biot-Savart law above.
%If we consider a filament that is infinitesimally thin, we find an energy $E = \rho \Gamma^2 R/2$ and a velocity of $v = \Gamma / 4 \pi R$.}

If we imagine a single vortex ring, we can define a vortex bubble as its region of influence within the surrounding fluid. 
If we take a two-dimensional slice of this ring as shown in Figure~\ref{fig:V_B}, we may represent the vortex bubble as the region influenced by two infinitely long, parallel vortex filaments with opposite vorticity. 
There is an interesting analog to vortex rings and current-carrying loops in the theory of electromagnetism. 
With a current-carrying wire, a dipolar magnetic field will be created and there is a well-defined spherical region in space where the loop's field overrides any external field. 
In this way, the current density within the wire is related to the vorticity of a vortex loop at its core and the wire's magnetic field is related to the fluid velocity, $v$~\cite{Fetter1966}.

ADD FIGURE

In the case of multiple interacting vortex rings, we can expect to find many similar features in superfluids to what has been found previously in classical, viscous fluids. 
If two vortex rings are generated in the same plane and in close proximity, it could be possible for the two velocity fields to interact, causing one ring to expand and slow down while the other contracts and speeds up. 
Under the right conditions, the lagging ring can pass the forward ring through a process known as ``leapfrogging"~\cite{Sommerfield1950, Caplan2014}.

In addition to leapfrogging, vortex rings can interact through direct collisions~\cite{Shariff1992}. 
In superfluid $^4$He, some of the earliest experiments on vortex collisions with vortex rings were performed by Schwarz in 1968~\cite{Schwarz1968}.
In the case of a head-on collision, two identical, moving vortex rings will first grow in size before dispersing into a series of smaller vortex rings around their common circumference~\cite{Lim1995}. 
These smaller rings are created by vortex reconnection, which can occur any time vortex lines are facing anti-parallel directions.

Finally, we will briefly discuss vortex reconnections.
As predicted by Feynman in 1955, vortex reconnections in a superfluid lead to larger vortices continually reconnecting into smaller ones until the loops become small enough to decay from dissipation or from interactions with boundaries.~\cite{Feynman1955}.
These reconnections are thought to produce sound waves when vortices directly interact and Kelvin waves when vortices indirectly interact~\cite{Paoletti2011}.
When a vortex ring structure is not pinned by either gauge fields or rotation, it will evolve naturally by reconnecting into smaller and smaller vortex rings~\cite{Jackson1999}. 
This means that we expect to see vortex reconnections in any sufficiently complicated vortex tangle~\cite{Barenghi2014}.
In addition, when simulations of vortex rings were performed by Jones and Roberts, they found that by computing vortex rings towards the limit of $R \rightarrow a$, they eventually found a solitary wave~\cite{Jones1982, Berloff2005}.


\section{Generation of gauge fields via the evanescent field of an optical nanofiber}

\section{The scissors mode with a vortex present}

\section{Generation, control and detection of 3D vortex structures in superfluid
 systems}
