\chapter{Generation, control and detection of 3D vortex structures in superfluid systems}
\label{ch:3d}

In Chapters~\ref{ch:1d} and~\ref{ch:2d}, we discussed various methods for quantum engineering involving one and two dimensional systems, respectively.
In this chapter, we will discuss specific implementation details for three-dimensional vortex generation and provide an example through which we may control three-dimensional vortex structures in BEC systems by using the artificial magnetic field generated by optical nanofibers.

\section{Vortex highlighting via Sobel filter}

As described in Chapter~\ref{ch:vortex}, vortex tracking in two dimensions is not always straghtforward for non-harmonic traps.
In three dimensions, vortices are no longer confined to a plane and can extend in any direction, so long as the vortex lines either end at the end of the superfluid or reconnect in the form of vortex rings or more complicated vortex structures.
This is a much more difficult problem which does not have many solutions in superfluid simulations where the superfluid does not fill the simulation domain.

The current state-of-the-art solution has been proposed by Villois \textit{et. al}, and required finding density dips in the superfluid as initial guesses as to where a vortes might exist.
From there, a vorticity plane is determined and the entire vortex is discovered by moving perpendicularly to the vorticity plane at each gridpoint.
This is a tedious and time-consuming process that does not lend itself well to GPGPU computation without largescale communication between the host and device, which is not recommended.
As such, we are currently seeking a more computationally efficient method for tracking vortices in three dimensions, and a possible method will be further discussed, but not tested in the conclusion of this work.

As our system do not necessarily fill the contents of our simulation domain, the proposed method will not work without some modification.
We could still use the method if we have some understanding of the trapping geometry; however, this is not always the case, as we will see in a future example simulation in Chapter~\ref{ch:vortex_states}.

As such, instead of focusing on vortex \textit{tracking}, we have instead implemented a simple vortex \textit{highlighting} scheme for three dimensions.
This is simply a Sobel filter on the condensate density, and can easily create crisp visualizations like those found in the computer graphics literature~\cite{guo2018}.

\jrs{describe edge detection, maybe canny?}

This is a difficult problem that required further study; however, vortex highlighting is enough for most three dimensional vortex simulations.
A possible method to track vortices by using a sobel filter can be found in the conclusion.

\section{CuFFT routine for gauge field simulation}

