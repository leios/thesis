\chapter{Controlling one-dimensional quantum systems}
\label{ch-1d}

We live in an inherently three-dimensional world, and most objects we interact with are similarly three dimensional.
Even so, theory cannot always describe this reality and there are several occasions where reducing the dimensionality of a system will allow for a better understanding.
For quantum simulations, many one and two dimensional approximations can be made in the laboratory, thus allowing for important physics to be understood with fewer data analysis techniques.

There are also many methods that can be used theoretically and computationally to allow for the generation of specific quantum states.
The act of generating these states quickly and efficiently is sometimes referred to as \textit{quantum engineering} and many of these methods have been use in experimental and theoretical studies [CITE REVIEWS of STA + EXP STUFF].
Though many of the quantum engineering schemes do not rely on the dimensionality of the system, some are best described in one dimension before being extended to two or three dimensions later.

In fact, because many one-dimensional systems can be generated in the experimental laboratory, there is a lot of interest in quantum engineering even from a one dimensional perspective.

In this section, we will motivate a particular set of one dimensional simulations by providing a method for the generation of largescale superposition states with highly-interacting bosons in a Tonks--Girardeau gas by using two techniques in quantum engineering: Quantum Optimal Control and Shortcuts to Adiabaticity.

\section{NOON states in Tonks--Girardeau gas}

\section{Shortcuts to adiabaticity}
\section{Optimal control}


\section{Nonadiabatic generation of NOON states in a Tonks--Garadeau gas}
