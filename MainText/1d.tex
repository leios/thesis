\chapter{Quantum engineering for one-dimensional quantum systems}
\label{ch:1d}

We live in an inherently three-dimensional world, and most objects we interact with are similarly three dimensional.
Even so, theory cannot always describe this reality and there are several occasions where reducing the dimensionality of a system will allow for a better understanding.
For quantum simulations, many one and two dimensional approximations can be made in the laboratory, thus allowing for important physics to be understood with fewer data analysis techniques.

There are also many methods that can be used theoretically and computationally to allow for the generation of specific quantum states.
The act of generating these states quickly and efficiently is sometimes referred to as \textit{quantum engineering} and many of these methods have been use in experimental and theoretical studies [CITE REVIEWS of STA + EXP STUFF].
Though many of the quantum engineering schemes do not rely on the dimensionality of the system, some are best described in one dimension before being extended to two or three dimensions later.

In fact, because many one-dimensional systems can be generated in the experimental laboratory, there is a lot of interest in quantum engineering even from a one dimensional perspective.

In this section, we will motivate a particular set of one dimensional simulations by providing a method for the generation of largescale superposition states with highly-interacting bosons in a Tonks--Girardeau gas by using two techniques in quantum engineering: Quantum Optimal Control and Shortcuts to Adiabaticity.

\section{Introduction to ultracold quantum systems}

As atomic systems are cooled to temperatures near absolute zero Kelvin, it becomes easier to discern their quantum properties which vary depending on the system.
Most elementary particles that make up atoms have intrinsic angular momentum known as \textit{spin}.
Electrons are elementary particles with a spin of $\frac{1}{2}$, and protons and nuetrons are made of elementary particles that sum to a spin of $\frac{1}{2}$.
The spin of the entire atom is determined by summing the constituent spin components.
If the sum results in an integer value, the atom is known as a \textit{boson}, and if the sum results in a half-integer value, the atom is known as a \textit{fermion}.

Because fermions have half-integer spin, they must obey the Pauli exclusion principle and are constrained to Fermi--Dirac statistics, which means their ground state will be composed of several fermionic pairs.
This creates a \textit{Fermi sea}, where particles fill defined energy levels from the bottom up with two particles per level.

On the other hand, bosons have integer spin and follow Bose--Einstein statistics and will condense into a single, macroscopic ground state when cooled~\cite{Einstein1925, Fetter2003}.
This state of matter is known as a Bose--Einstein Condensate (BEC) and it has the properties of a superfluid, this will be discussed fully in Chapter~\ref{ch:vortex}.

There are notable exceptions to these rules, such as the highly correlated Tonks--Girardeau gas where bosons may act as spinless, non-interacting fermions \cite{Girardeau}.
There also exists a regime where interacting fermions can also condense into a BEC-like system~\cite{Nozieres1985, Bulgac2014}, but this work focuses primarily on bosonic systems and will not discuss fermionic systems further.
Now we will extend this framework to two important quantum systems: the Bose--Einstein condensate and the Tonks--Girardeau gas, both of which are used heavily in this work.


\subsection{Bose--Einstein condensation and the Gross--Pitaevskii Equation}

As mentioned in Section~\ref{sec:introintro}, bosons in a BEC condense into the same (ground) state, meaning we must introduce a many-body Hamiltonian for the system and take inter-particle interactions into account.
For the purposes of this body of work, we will only consider two-body interactions and assume any interactions between three or more atoms are unlikely and negligible.
We may write the Hamiltonian with two body interactions in the second quantized form as
\begin{equation}
    \hat H = \int d\mathbf{r} \hat \Psi^\dagger(\mathbf{r})\left[-\frac{\hbar^2}{2m}\nabla^2 + V_0(\mathbf{r}) \right]\hat \Psi(\mathbf{r}) + \frac{1}{2} \int d\mathbf{r} d\mathbf{r'} \hat \Psi^\dagger(\mathbf{r}) \hat \Psi^\dagger(\mathbf{r'}) V(\mathbf{r} - \mathbf{r'})\hat \Psi(\mathbf{r'}) \hat \Psi(\mathbf{r})
    \label{eqn:2nd}
\end{equation}
where $\mathbf{r}$ and $\mathbf{r'}$ are the positions of the two colliding particles, $V(\mathbf{r}-\mathbf{r'})$ is the interaction potential, and $\hat \Psi^\dagger(\mathbf{r})$ and $\hat \Psi(\mathbf{r})$ are the creation and annihilation operators that follow the commutation relation, $[\hat \Psi(\mathbf{r}),\hat \Psi^\dagger(\mathbf{r})] = \delta(\mathbf{r} - \mathbf{r'})$.

In the case of a BEC at $T\approx0$, we may perform a Bogoliubov expansion~\cite{Bogoliubov1947, Dalfovo1999}
\begin{equation}
    \hat \Psi (\mathbf{r}, t) = \Phi(\mathbf{r},t) + \delta \hat \Phi(\mathbf{r},t)
\label{eqn:bog}
\end{equation}
where $\Phi(\mathbf{r},t) \equiv \langle \hat \Psi(\mathbf{r},t) \rangle$ is the wavefunction of the condensate known as the ``order parameter'' and $\delta \hat \Phi(\mathbf{r},t)$ represents fluctuations of the BEC system.
In a BEC, the condensate density is defined as
\begin{equation}
    n(\mathbf{r},t) = |\Phi(\mathbf{r},t)|^2.
\end{equation}

Now we may use the Heisenberg equation of motion
\begin{equation}
    i\hbar \frac{\partial}{\partial t}\hat \Psi(\mathbf{r},t) = [\hat \Psi, \hat H]
\end{equation}
    to determine the time evolution of the field operator $\hat \Psi(\mathbf{r},t)$ as
\begin{equation}
    \frac{\partial}{\partial t}\hat \Psi(\mathbf{r},t) = \frac{1}{i\hbar}\left[-\frac{\hbar^2}{2m}\nabla^2 + V_0(\mathbf{r}) + \int d\mathbf{r'} \hat \Psi^\dagger(\mathbf{r'}, t)V(\mathbf{r'} -\mathbf{r})\hat \Psi(\mathbf{r'},t)\right]\hat \Psi(\mathbf{r},t),
\end{equation}
which follows from Equation~\eqref{eqn:2nd} after integrating over $\mathbf{r}$. 
Now we must apply a few approximations for the BEC:
\begin{enumerate}
    \item In the Bogoliubov expansion, Equation~\eqref{eqn:bog}, we assume that $\delta \hat \Phi(\mathbf{r},t)$ is small at $T = 0\text{K}$, and thus $\hat \Psi(\mathbf{r},t) \approx \Phi(\mathbf{r},t)$. 
    \item Two bosons will only interact with a contact potential of the form
    \begin{equation}
        V(\mathbf{r'}-\mathbf{r}) = g\delta(\mathbf{r'} - \mathbf{r}),
    \end{equation}
    which has a strength given by
    \begin{equation}
        g \equiv \frac{4 \pi \hbar^2 a_s}{m},
    \end{equation}
    where $a_s$ is the species and state-dependent s-wave scattering length.
    %\item The bose gas is dilute and the inter-particle spacing is much larger than $a_s$.
\end{enumerate}

With these approximations, we may write the time-dependent Schr\"odinger equation as
\begin{equation}
    i\hbar \frac{\partial}{\partial t}\Phi(\mathbf{r},t) = \left( - \frac{\hbar^2}{2m} \nabla^2 + V_0(\mathbf{r}) + g |\Phi(\mathbf{r},t)|^2\right)\Phi(\mathbf{r},t).
\end{equation}
This equation is known as the nonlinear Schr\"odinger equation due to the presence of the $|\Phi(\mathbf{r},t)|^2$ term.
In the BEC community, the equation is usually called the Gross-Pitaevskii equation (GPE).
When written in the time-independent form it determines the chemical potential $\mu$ of the condensate system~\cite{Gross1961, Pitaevskii1961}
\begin{equation}
    \mu\Phi(\mathbf{r}) = \left( - \frac{\hbar^2}{2m} \nabla^2 + V_0(\mathbf{r}) + g |\Phi(\mathbf{r})|^2\right)\Phi(\mathbf{r},t).
    \label{eqn:GP}
\end{equation}

This equation allows us to determine the full dynamics of a BEC system and the numerical solutions will be discuss in Section~\ref{sec:code}. Similar derivations of the GPE can be found in many introductory texts on BEC physics, such as~\cite{Fetter2003,  Pethick2002, Fetter2009}.

\subsection{Tonks--Girardeau gas}

As discussed, the Tonks--Girardeau gas consists of a number of bosons that have the properties of spinless, non-interacting fermions.
This is a particular case of the one-dimensional GPE where $g\rightarrow\infty$.
In this case, the bosons cannot pass each other and cannot be at the same location, which acts formally similar to the Pauli-exclusion principle for fermionic sstems.
In this case, the Hamiltonian can be solved by the Bose--Fermi mapping theorem \cite{girardeau2001ground, girardeau2001measurement}, which replaces the interaction terms in the hamiltonian with a boundary condition on the many-body bosonic wavefunction.

\begin{equation}
\Psi_B(x_1, x_2, \ldots, x_N) = 0,\qquad \mathrm{if}\qquad x_i - x_j = 0 \quad\textrm{with}\quad i \ne j.
\end{equation}

The Bose--Fermi mapping theorem allows us to replace strongly interacting bosons by spinless, non-interacting fermions, on which we can use the Slater determinant \jrs{CN},

\begin{equation}
\Psi_F (x_1, x_2, \ldots, x_N) = \frac{1}{\sqrt{N}} \det\Big[\psi_n(x_j)\Big]_{n,j=1}^N.
\end{equation}

\noindent where $\psi_n(x_j)$ are the single-particle eigenstates of the trapping potential $V_0$.
Because the Fermionic many-body wavefunction is antisymmetric, this needs to be symmetrerized for bosonic states as, 

\begin{equation}
\Psi_B(x_1, x_2, \ldots, x_N) =
\prod_{i < j}
\mathrm{sgn}(x_i - x_j)\Psi_F(x_1, x_2, \ldots, x_N)
\end{equation}

\noindent which means that calculating the time evolution of a Tonks--Girardeau gas requires evolving single-particle states, governed by a simple Hamiltonian.


\section{NOON states in Tonks--Girardeau gas}

\section{Shortcuts to adiabaticity}
\section{Optimal control}


\section{Nonadiabatic generation of NOON states in a Tonks--Garadeau gas}
