\chapter*{Introduction to quantum systems}

The Split-Step Fourier Method (SSFM) is an essential technique for simulating a large variety of physical systems; however, for the purposes of this text, we will be focusing on its application to quantum systems and will use primarily physical arguments to understand the details of the method, itself.
As such, it is important to discuss several important principles of quantum systems, incuding the Schr\"odinger equation which governs the dynamics for simple quantum systems.

\section{The Schr\"odinger equation}
Quantum particles are often described by their wavefunction $\Psi(\mathbf{r},t)$, where $\mathbf{r}$ is a position-space variable and $t$ is time.
This does not have a simple physical interpretation; however, the wavefunction density $|\Psi(\mathbf{r},t)|^2$ can be interpreted as a probability distribution in position-space, where peaks represent areas where the quantum particle is likely to be found.
This means that

$$
\int_\infty^\infty |\Psi(\mathbf{r},r)|^2 d\mathbf{r} = 1,
$$

which is a common principle to use to ensure the system is normalized correctly.

For the purposes of this body of work, we can simulate the dynamics of a quantum system by solving the Schr\"odinger equation,

$$
i\hbar\frac{\partial\Psi(\mathbf{r},t)}{\partial t} = \left(\frac{p^2}{2m} + V_0\right) \Psi(\mathbf{r},t)
$$

where $p = -i\hbar\nabla$ is the momentum operator and $V_0$ is the trapping potential.
For the case of the simple harmonic oscillator in one dimension, $V_0 = \omega_x x^2$, where $\omega_x$ is the trapping frequency in the $x$-dimension.
In this case, the dynamics can be described almost entirely by the right-hand side of the equation and the Hamiltonian operator,

$$
\mathcal{\hat H} = \frac{p^2}{2m} + V_0
$$

which noticeably has two separate operators: one in momentum-space ($H_p = \frac{p^2}{2m}$) and another in position-space ($H_v = V_0$).
In this work, we modify the Hamiltonian to match the various quantum systems we would like to simulate

Now, we will extend this framework to two important quantum systems: the highly correlated Tonks--Girardeau gas and the Bose--Einstein condensate, both of which are used heavily in this work.

\section{Tonks--Girardeau gas}

Discuss Bosons / Fermions

\section{Bose--Einstein condensation}

Discuss BEC
