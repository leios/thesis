\section*{Introduction}
The Split-Step Fourier Method (SSFM) is an essential technique for simulating a variety of physical systems and is particularly useful for simulating the propagation of wave packets for single and multimode fiber systems~\cite{agrawal2000} and various quantum systems \jrs{CN}, including all simulations performed in this work.
For the purposes of this text, we will be focusing on its application to quantum systems and will use primarily physical arguments to understand the details of the method, itself.
We will also discuss several numerical techniques for optimally simulating quanum systems on Graphics Processing Units, along with applications software developed for this purpose: GPUE, the Graphics Processing Unit Gross-Pitaevskii Equation Solver.

As such, it is important to first discuss several principles of quantum systems in Chapter~\ref{ch:qs}, incuding the Schr\"odinger equation which governs the dynamics for simple quantum systems along with several notable variations that correspond to physical systems to be discussed in this work.
Further elaboration on advances in scientific computing, including the massively parallel Graphics Processing Unity (GPU) will be discussed in Chapter~\ref{ch:gpu}, and the SSFM method, along with variations on the method for vortex simulations will be discussed in Chapter~\ref{ch:splitop}.
In Chapter~\ref{ch:1d}, we will discuss the basics of quantum engineering, by generating macroscopic superposition states of highly-interacting quantum particles with quantum optimal control and shortcuts to adiabaticity.
In Chapter~\ref{ch:dynamics}, we will discuss specifically superfluid simulations and several notable methods for generating and controlling vortex structures both theoretically and experimentally, along with numerical hurdles with simulating such systems on GPU architecture, and in Chapter~\ref{ch:2d}, we will discuss an application of the GPUE codebase in the generation of chaotic vortex dynamics with a small number of vortices in a two-dimensional superfluid simulation.
In Chapter~\ref{ch:3d}, we will discuss challenges in simulating fully 3-dimensional, multi-component superfluid systems on GPU architecture, including dynamic field generation, storage limitations, and spectral methods to optimize the simulation, itself.
In Chapter~\ref{ch:vortex_states}, we will apply the numerical techniques introduced in this text to simulate a novel device that couples the evanescent field of an optical nanofiber to generate vortex ring-like structures in a toroidally trapped superfluid system.
Finally, we will conclude by discussing key results from this work, along with future directions in both theoretical and computational physics.
