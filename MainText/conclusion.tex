\chapter*{Conclusion}
\label{ch:conclusion}

\section{Further development of GPUE}

Though the GPUE codebase is roughly feature complete, there are several directions for future development.
In particular, GPUE can potentially incorporate a novel vortex-tracking scheme for two and three-dimensional simulations.
In addition, there are several performance optimizations that can be done by implementing a multi-GPU transpose, as this would eliminate the issue with coalescence with FFT's across mutliple GPU's and make larger-scale simulations possible.
The back-bone of GPUE could also be used for a general-purpose Hamiltonian solver by using the AST implementation on GPU devices to easily parse the Hamiltonian into individual components for simulation.
In addition, due to engineering constraints, it is not currently possible to fund a full-time developer for the GPUE codebase.
As such, it is important to discuss the future maintainability of GPUE and its utility in for future simulations for quantum systems.

\subsection{Vortex tracking in $n$ dimensions}

The GPUE codebase currently has the capability of tracking vortices in two dimensions and highlighting vortices in three; however, vortex tracking has yet to be implemented as there are no reliable and general methods for tracking three dimensional vortex structures in superfluid simulations.
In addition, the vortex tracking method in GPUE is currently unstable for non-harmonic traps in two dimensions, and in many cases, the user does not know the precise geometry of the trapping system.
As such, a generalized vortex tracking methods for $n$-dimensional simulations is desired.

In 2016, a method for three dimensional vortex tracking was proposed by Villois \textit{et. al.}~\cite{villois2016}; however, this method has no computational complexity bound, assumes periodic boundary conditions, and required a large amount of communication between the device and host.
We have considered development of a similar method that leverages our vortex highlighting scheme to create $n$-dimensional vortex skeletons for vortex tracking in two and three dimensions.
\jrs{ADD MORE IF KEEPING}

\subsection{General purpose Hamiltonian solver}

\subsubsection{GPUE.jl}
After considering our options, we have begun development of GPUE.jl, the a julia re-write of GPUE.

\subsection{Future simulations of quantum systems}

In general, there are many new types of simulation possible with the gPUE codebase.
One particularly interesting direction is in the area of three-dimensional multicomponent superfluid simulations.
\jrs{Heat engines?}
