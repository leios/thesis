\chapter*{Conclusion}

\section{Overall conclusions}

\label{ch:conclusion}

This thesis has presented my efforts to create a massively parallel SSFM codebase for the simulation of superfluid vortex dynamics in Bose--Einstein Condensates (BECs).
Starting from the SSFM and a discussion on the dynamics of ultracold atomic systems, I motivated the dynamic quantum engineering by introducing quantum optimal control and shortcuts to adiabaticity.
Both of these methods were used to show that it is possible to generate macroscopic superposition states in a Tonks--Girardeau gas when on a ring with a barrier to break rotational symmetry.
From there, I introduced the GPGPU and the GPUE codebase, emphasizing existing challenges in the field and the methods I used to overcome them.
In the process, I also briefly mentioned the challenging problem of memory coaliescence when using spectral methods on GPU hardware and proposed an additional software package as method to further optimize the FFT operations with a distributed, multi-GPU transpose.
After this, I introduced an example physical system that showed it is possible to generate chaotic vortex dynamics in few-vortex systems and emphasized the need for vortex tracking methods and post-processing methods, by calculating the Lyapunov exponent on the vortex trajectories.
Finally, I introduced a three-dimensional example system that generates, controls, and detects vortex ring-like geometries in a toroidally-trapped BEC coupled to the artificial magnetic field generated by an optical nanofiber.

Throughout this work, there are several physical and computational areas that require further development in the future and these will be further discussed in this section.

\jrs{Not really sure which future directions to highlight here...}
\section{Further development of GPUE}

Though the GPUE codebase is roughly feature complete, there are several directions for future development, many of which were described in Chapter~\ref{ch:gpu}.
In particular, a re-write of GPUE in julia along with developing an $n$-dimensional, distributed, GPU transpose are currently being worked on.
The former will allow for GPUE to be more maintainable and require less development time in the future.
The latter is applicable to a wide range of spectral methods and might allow for several methods to become relevant again on HPC environment.
As both of these were discussed at length in Chapter~\ref{ch:gpu}, I will not discuss them further here; however,
there are future directions where proper development has not begun, such as new vortex tracking methods and potentially using expression trees for general-purpose Hamiltonian solutions.

\subsection{Vortex tracking in $n$ dimensions}

The GPUE codebase currently has the capability of tracking vortices in two dimensions and highlighting vortices in three; however, vortex tracking has yet to be implemented as there are no reliable and general methods for tracking three dimensional vortex structures in superfluid simulations.
In addition, the vortex tracking method in GPUE is currently unstable for non-harmonic traps in two dimensions, and in many cases, the user does not know the precise geometry of the trapping system.
As such, a generalized vortex tracking methods for $n$-dimensional simulations is desired.

In 2016, a method for three dimensional vortex tracking was proposed by Villois \textit{et. al.}~\cite{villois2016}; however, this method has no computational complexity bound, assumes periodic boundary conditions, and required a large amount of communication between the device and host.
We have considered development of a similar method that leverages our vortex highlighting scheme to create $n$-dimensional vortex skeletons for vortex tracking in two and three dimensions.
\jrs{ADD MORE IF KEEPING}

\subsection{General purpose Hamiltonian solver}

\subsubsection{GPUE.jl}
After considering our options, we have begun development of GPUE.jl, the a julia re-write of GPUE.

\section{Future simulations of quantum systems}

In addition to further developments of GPUE, there are also several new simulations that can be performed now on GPU hardware, such are multicomponent simulations with gauge fields and dynamic studies of the system introduced in Chapter~\ref{ch:vortex_states}.
Because GPUE allows for the simulation of dynamic variables through expression trees, further three-dimensional STA studies can also be performed.

Ultimately, this work has provided a toolbox for the simulation of various quantum phenomenon that were computational inctractable before now, including the three-dimensional simulations of superfluid turbulence without relying on vortex-filament mthods, and simulations of multicomponent systems with gauge fields.
It has also developed novel methods for maximizing the size of the simulated domain with the SSFM, along with tools like the DistributedTranspose.jl that allow for spectral methods to be more widely used in HPC environments.
